我们要作图,或为写文章写书排版,了解纸张的尺寸自然是有用的。国际标准ISO 216的纸张尺寸可分为A、B、和C三种系列,其制定都跟数学有关。
在中国和欧洲,打印文件纸张的比较常用标准尺寸是A系列中的 A4,即,210毫米\texttt{x}{ }297毫米 ($\approx$ 8.27 英寸 x 11.69 英寸)。
 德国科学家利希滕贝格(Georg Christoph Lichtenberg) 发现高宽比为$\sqrt{2}$
的矩形具有有趣的特点。
设一张矩形的纸宽为$w$,高为$h$,竖着看我们假设$w <h$。横着在高度一半处切裁为两个全等的更小矩形。小矩形不管是顺时针还是逆时针旋转 $90^\circ$,竖起来后设新的宽为$w'$,高为$h'$。则
$w' = h/2$ 且 $h' = w$。若使原矩形与新的小矩形相似,则两个高宽比须相等:
$$\frac{h}{w} = \frac{w}{h/2}.   $$
于是,$h^2= 2 w^2$, 或
 $$h= \sqrt{2} w.$$
这里我们又遇到了一个无理数, $\sqrt{2}$。在工程上,我们较真的程度有赖于技术。
	
\begin{kaishu}习题.\end{kaishu} 执行如下 Python 代码,确认记忆中的$\sqrt{2}$值 $1.4142135623730951\cdots$。
\begin{spacing}{0.8}
	\begin{small}
	\begin{lstlisting}[language=Python]
print("2 的平方根 = ", 2**.5)
\end{lstlisting}
\end{small}
\end{spacing}

A 系列不同开本的纸张从一平方米(1,000,000 平方毫米)的A0开始。设宽为$w_0$ 毫米,则高为
$\sqrt{2}w_0$ 毫米。于是$\sqrt{2}w_0^2 = 1000000$。
由此,
$w_0 = \sqrt{1000000/\sqrt{2}}=1000/\sqrt[4]{2}, 
h_0=\sqrt{2}w_0=1000\cdot\sqrt[4]{2}$。
几行Python代码可算出 A 系列不同开本纸张的尺寸。注意裁开一次页数加倍,所以A$n$是$2^n$开,A4是16开($2^4=16$)。

\vspace{.4cm}
\begin{spacing}{0.8}
	\begin{small}
	\begin{lstlisting}[language=Python]
#%% A系列纸张的理论尺寸
w = 1_000 / 2**.25 # 下划线`"\_"`分隔仅为阅读方便,无实际编程作用
h = 1_000 * 2**.25 # 2**.25`$=2^{.25}$`是2的4次方根即2的0.25次幂
print('A系列纸张的理论尺寸(mm x mm):')
for i in range(0, 11):
  # {:>3} 表示3个字符向右对齐。宽高四舍五入到整数。
  # {:3.0f} 浮点数只显示3位整数,自动向右对齐。
  # {:<4.0f} 浮点数只显示4位整数,向左对齐。
  print("{:>3}: {:3.0f} x {:<4.0f}".format(
    'A'+str(i), w, h), end='   ')
  # {:6.2f}: 包括小数点,浮点数展示6个字符向右对齐,小数点后面保留2位数。
  # {:<7.2f}: 包括小数点,浮点数展示7个字符向左对齐,小数点后面保留2位数。
  print(" {:6.2f} x {:7.2f}".format(w, h), end='  ')
  # 用 int() 舍弃小数部分,宽高仅保留整数部分。
  print("   {:3.0f} x {:4.0f}".format(int(w), int(h)))
  tmp = w  #暂存原宽度备作下一个高度。
  w = h/2   #新宽度为原高度的一半。
  h = tmp   #新高度为原宽度。
\end{lstlisting}
\end{small}
\end{spacing}
\vspace{.4cm}\label{a_paper_py}

输出结果如下:
\vspace{.4cm}
\begin{spacing}{0.8}
%	\begin{small}
A系列纸张的理论尺寸(mm \textsf{x} mm):

\begin{tabular}{l l l l}
 A0: & 841 \textsf{x} 1189  &  840.90  \textsf{x} 1189.21   &  840  \textsf{x} 1189 \\
 A1: & 595 \textsf{x} 841 &    594.60  \textsf{x} 840.90   &   594  \textsf{x} 840\\
 A2: & 420 \textsf{x} 595   &  420.45  \textsf{x} 594.60  &    420  \textsf{x} 594\\
 A3: & 297 \textsf{x} 420   &  297.30  \textsf{x} 420.45   &   297  \textsf{x} 420\\
 A4: & 210 \textsf{x} 297   &  210.22  \textsf{x} 297.30    &  210  \textsf{x} 297\\
 ...
\end{tabular}


\end{spacing}
\vspace{.4cm}\label{a_paper_py}
国际标准 ISO 216 中毫米仅保留到整数。第一列结果四舍五入显然不符合切裁纸张的实际。第三列用 \texttt{int()}丢弃小数部分,宽高仅保留整数部分,所以应该最符合实际。事实上,除了ISO 216 A1的尺寸是594毫米\texttt{x}{ }841毫米而不同外,第三列的其它都符合国际标准。

\begin{kaishu}练习.\end{kaishu} 在上面的Python代码中的,把\texttt{for} 循环句改为\texttt{while}循环句,并且使输出加一列说明开数。

B系列纸张仍然遵循高宽比约为$\sqrt{2}$的原则,但定义B0的宽为1000毫米(1米),高为1414毫米.

\begin{kaishu}练习.\end{kaishu} 写出 Python 程序,计算 B0 到B10 的开本和理论尺寸。 

采用国际标准中不同尺寸的纸张,由于高宽比例固定为$\sqrt{2}$,除了切裁加开在工程意义上不改变高宽比例外,不同系列和开本的文件可以直接缩放影印而不会造成纸面图案有边缘裁切的问题。同系列纸张的缩放倍数更是2的幂.

中国书籍,查看菲页背后有关印刷信息的一页,一般会见到纸张尺寸的字样.
比如:开本 787x1092 1/32, 850x1168 1/32, ... 
787x1092 是 787毫米x{ }1092毫米(约31英寸x{ }43英寸)的正度纸,高宽比例约为1.37.
850x1168 是 850毫米x{ }1168毫米的大度纸,高宽比例约为1.39.
1/32 则是整幅纸裁为32张.
它们跟国际标准不一致,大概会被逐步淘汰.

美国纸张尺寸标准也不按$\sqrt{2}$的高宽比例。 常用标准信纸的尺寸是 8.5 英寸\texttt{x}{ }11 英寸($\approx$ 215.9 毫米 x 279.4 毫米) 。法律用纸尺寸为8.5 英寸\texttt{x}{ }14 英寸($\approx$ 215.9 毫米 x 355.6 毫米)。美国是为数不多的不采用公制而采用英制的国家。尤为麻烦的是,计算机网络和软件方面,仍然时不时要同英寸打交道:1英寸=2.54厘米,或者$1\,'' = 2.54$ cm.

\begin{Exercises}
	\item 在仍然常见的书籍印刷 787x1092 毫米和 850x1168 毫米纸张系列中, 16 开本每页的理论尺寸是多少? 32 开本每页的理论尺寸是多少? 找本所说规格的书测量一下,验证你的答案. 如果答案不符, 为什么?
\end{Exercises}

