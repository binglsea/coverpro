\chapter*{序\quad 言}

\setcounter{page}{1}
\thispagestyle{empty}
\markboth{序\quad 言}{序\quad 言}
\addtocontents{toc}{\vspace{3mm}\leftline{\bf\large 序言}\vspace{3mm}}

本书介绍基本覆盖性质的特征理论,重心是仿紧空间和次亚
紧空间的刻画.
所谓基本覆盖性质还包括亚紧性,次仿紧性,
正规覆盖,可缩性和強仿紧性等.
某类覆盖性质的特征或
刻画(characterizations)是指与该性质的原始定义等价的命
题.
这些特征命题一般比原始定义更弱.
自从A.H.Stone [1948]
建立$T_2$仿紧性的重合定理以来,寻找仿紧空间及广义仿紧空间
的各种各样的特征的努力一直延续至今.
发现不少美妙定理与精巧技术,形成系统理论.
本书试图叙述这一领域的主要成果,全部给予证明.
仿紧空间的刻画分别在$\S$\ref{ch2.3} 与
$\S$\ref{ch4.2} 介绍, $\S$\ref{ch2.3} 介绍不附加分离公理的仿紧性的刻画.
$\S$\ref{ch4.2} 则介绍包含$T_2$仿紧空间为子类的$\lambda$完满正规空间的刻画.
作为它的另一子类,正规$\lambda$强仿紧空间,在$\S$\ref{ch4.3} 中介绍.
$\S$\ref{ch2.1} 介绍次亚紧空间的刻画.
这三节是特征理论的重心. $\S$\ref{ch3.1} 介绍
正规覆盖的刻画,包含了点集拓扑学发展早期的一些好结果.
$\S$\ref{ch3.2} 介绍集体正规空间与可缩空间的刻画.
这两节的内容是基本的, 不仅有其自身的意义,也是$\S$\ref{ch4.2} 中定理证明需要引用的.
$\S$\ref{ch4.1} 与$\S$\ref{ch2.2} 分别介绍次仿紧与亚仿紧空间的刻画.
每种覆盖性质有一类型刻画,我们称之为分解定理,即这种刻画表现
为比它较弱的两种拓扑性质之和(或两类较弱空间类的交).
其中一类是可扩型空间,另一类我们称之为弱覆盖性质.
我们在$\S$\ref{ch1.2} 与$\S$\ref{ch1.3} 中分别介绍这些空间需要的知识.其中也有值得学
习和欣赏的好结果。

有关覆盖性质的特征理论的已有文献,我愿推荐: [Bur84], [Jun80], [Gao2008] 和 [Yas89].

本书假定读者了解点集拓扑学的基础知识,如 [Eng77] 的前5章,或
[Gao2008]的前6章.集合论只需了解基数与序数的一般性质.

本书的参考文献只限于本书所引用者.我在此向每一位作者谨致敬意和谢忱.最后, 我对部分作者被引用的工作 (他们引入的概念和建立的定理) 做了一个索引.
按照每一位作者被引用论文的最早发表年份为序
排列他们的姓名.这个索引有助于了解特征理论的发展历史.

{\kaishu
	\begin{center}
		\hspace*{88mm}蒋继光\\
		\hspace{88mm}于成都四川大学竹林村
		\hspace*{88mm}乙未暮春
	\end{center}
}


