\documentclass[main.tex]{subfiles}
\begin{document}
	
	覆盖性质有一类型的刻画是把它们分解成另两类较弱的拓扑性质的和.
	其中一类是可扩型性质,另一类是较弱的其它覆盖性质.
	例如,一个拓扑空间是仿紧的当且仅当它是可扩的和次亚紧的.
	我们把这种类型的刻画叫分解定理.本章第2节介绍用于各种
	分解定理的弱覆盖性质.第3节则介绍用作另一分解因子的可扩型性质.
	它们的应用将在后面的章节中介绍.
	
	
	\section{记号、术语与基本事实}
	集合简称\begin{kaishu}集\end{kaishu}\index{集, set},集的元素
	简称\begin{kaishu}元\end{kaishu}\index{元, element}.本书中的全称量词“对每一个”
	或符号“$\forall$”经常省略.倒如, “对每一个
	$\alpha\in\Delta$, $A_\alpha \subset B_\alpha$”,
	常简述为“$A_\alpha \subset B_\alpha$”.
	但存在量词$\exists$不可省略.等号
	“=”的基本用法是,它两端的集具有相同的元.
	我们还赋与它一种广义的用法, 让$P, Q$表示由若干字母组成的符号,则
	$P=Q$可以表示$P$是$Q$的一个名称(暂用的或专用的).
	例如$A=\{x,y\}$,
	
	我们用$\alpha, \beta, \gamma$等表示序数. $\alpha = \{\beta: \beta < \alpha\}$. $\beta \in \alpha \Leftrightarrow\beta<\alpha$. 基数是
	初始序数,用$\kappa,\lambda$ 等表示. $\omega = \{0, 1, 2, \dots\}$ 表
	最小无限基数.它的元,即自然数,用$m, n, i , k$等表示.
	对$n>0, n = \{0,1,2, \dots, n-1\}$. 我们用$|A|$表示$A$ 
	的\begin{kaishu}势\end{kaishu}\index{势, power}或\begin{kaishu}基数\end{kaishu}\index{基数, cardinal}.
	记$[A]^0=\{\phi\}$. 对$n\ge1, [A]^n = \{S\subset A: |S| = n\}$.
	$[A]^{<\omega} = \bigcup_{n<\omega}[A]^n$.
	
	\section{弱覆盖性质}
	\section{可扩型空间}
	
\end{document} 
