\section{可扩型空间}\label{ch1.3}
本节介绍各类可扩型性质,包括集体正规的概念和所需的性质.
它们主要应用于各类覆盖性质的分解定理.集体正规空间是 Bing[1951]为了研究空间的可度量性而引入的. Kat\v{e}tov [1958]的研究首先涉及可扩性质.可扩空间的定义和系统研究则始于
Krajewski[1971].

\begin{definition}
	\textnormal{(i)} 设$\mathscr{F}=\{F_\alpha:\alpha\in\Delta\}$是$X$的一个子集族.
	$X$的子集$\mathscr{G}$称为$\mathscr{F}$的一个\underline{扩张},如果
	$\mathscr{G}=\{G_\alpha:\alpha\in\Delta\}$使得$\forall\alpha\in\Delta$, $F_\alpha\subset G_\alpha$. 一个扩张称为\underline{开的}, 如果它的每个元是开集.
	
	\textnormal{(ii)} 设$\lambda\ge 2$, $X$ 是 \underline{$\lambda$集体正规的},如果$X$的每个势$\le\lambda$的离散闭子集族的有一个离散的开扩张.
	
	\textnormal{(iii)} 设$\lambda\ge \omega$, $X$是 \underline{$\lambda$可扩的}
	\textnormal{(}\underline{$\lambda$ 几乎可扩的}\textnormal{)},如果$X$的每个势$\le\lambda$的局部有限子集族有一个局部有限的\textnormal{(}点有限的\textnormal{)}开扩张.
\end{definition}

\begin{note}
	显然, $X$是$\lambda$可扩的$\Leftrightarrow X$内的每个势$\le\lambda$的局部有限闭子集族有一个局部有限的开扩张.
\end{note}

\begin{fact}
	\textnormal{(i)} 	$X$是$\lambda$集体正规的$\Leftrightarrow X$内的每个势$\le\lambda$的离散闭子集族有一个非交的开扩张.
	
	\textnormal{(ii)} $X$是$\lambda$可数集体正规的$\Leftrightarrow X$是正规的.
\end{fact}

证明. (i), (ii)可分别参见[Eng77]的 $\S$5.1.17以及$\S$2.1.14. $\square$

\begin{definition}
	设$\lambda\ge \omega$.

	\textnormal{(i)} $X$是 \underline{$\lambda$有界可扩的}
\textnormal{(}\underline{$\lambda$ 有界几乎可扩的}\textnormal{)},如果$X$的每个势$\le\lambda$的有界局部有限闭集族有一个局部有限的\textnormal{(}点有限的\textnormal{)}开扩张.

	\textnormal{(ii)} $X$是 \underline{$\lambda$离散可扩的}
\textnormal{(}\underline{$\lambda$ 集体几乎正规的}\textnormal{)},如果$X$的每个势$\le\lambda$的离散闭集族有一个局部有限的\textnormal{(}点有限的\textnormal{)}开扩张.
\end{definition}

这里的集体几乎正规空间在[Boo73]中称为点式集体正规的.

(ii) 中的概念出自[SK71], 有界可扩性出自[Kat\v{e}58].