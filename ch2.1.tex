\documentclass[main.tex]{subfiles}
\begin{document}
本章介绍仿紧与次亚紧空间的刻画,且不附加分离公理.无论从内容或方法来说, $\S$\ref{ch2.1} 和 $\S$\ref{ch2.3} 是本书的主要部分.

%-----------
\section{次亚紧空间}\label{ch2.1}
次亚紧空间的主要刻画定理是Junnila[1978]建立的.它是覆性质特征理论的重要工作.
本节前部分介绍这一工作.后部分介绍作者提出的次亚紧空间的分解定理
和与它密切相关的集体$\theta$正规空间的刻画定理.它们是 Jiang[1988]工作的改进.

\begin{definition}
	设$\mathscr{U}=\{U_\alpha:\alpha\in\Delta\}$, $\mathscr{V}, \mathscr{V}_n, n<\omega$是$X$的开覆盖.
	
	\textnormal{(i)} 称$\mathscr{V}$在$x\in X$处\underline{点式$w$ \textnormal{(}点星形
	$\dot{F}$\textnormal{{)加细$\mathscr{U}$}}}
\end{definition}	









\end{document} 
